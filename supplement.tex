\subsubsection{Web presence}
Tools vary in terms of web presence. For example, some tools are web-based, some tools simply rely on a README file in a code repository, with simple installation information, while others provide extensive information and documentation on a separate website. Google Analytics can allow for more fine-grained tracking of how users are interacting with web-based tools or websites and can be helpful for later assessments, in addition to improving access, as one can also perform search engine optimization (SEO) to further encourage the visibility. For example, a developer could track what parts of the documentation users appear to read more frequently and pair this with information about common errors. For web-based tools, great detail can be determined about user interaction. Again Google Analytics can be helpful or for tools using a web-server manual log file inspection, or a service like Cronitor \cite{cronitor}  if the tool relies on a web-based server using cron \cite{cron_2009} job scheduling can also be informative.  It is however important to be mindful of user privacy regulations, see section \ref{sec:legal_ethics}.

\subsubsection{Citability}
Another component that can assist with evaluations is providing users with a method of citation, as well as information about how to cite the software, as some users may not be familiar with such practices. This is often done by publishing a manuscript about the software in a scientific journal and providing information about the publication on the software website and code repository. In addition, digital object identifiers (DOI), from publishing platforms like Zenodo \cite{zenodo}, can be used for other less conventionally cited materials, such as documentation, case studies, data, and the actual software itself.

Services like Altmetric \cite{noauthor_altmetric_2015}, allow for deeper analysis of engagement for anything with a DOI. They provide reports and badges that can be added to websites or manuscripts (depending on where they are published) that indicate how often a DOI is cited in multiple sources besides scientific articles, such as blogs, news articles, Wikipedia, social media, and more.  Individuals or organizations can use Altmetric to get such reports by simply searching for the DOI. These reports also have links to individual social media posts, citing articles, and more. In addition, Semantic Scholar \cite{noauthor_semantic_nodate} provides reports that indicate where citations have occurred within scientific articles. 


\subsubsection{Documentation}
Documenting software not only helps guide users but can also enable collection of useful metrics.  Documentation on websites can be tracked and provide detailed information about such usage. If a particular page has more traffic, it may indicate that the aspect of the tool discussed on that page is either particularly popular or particularly confusing. As with all metrics, pairing information together can aid interpretation. Providing different types of documentation is another step that assures confidence in software, and enables varying roles to succeed in using the software. For example, command-line usage is important for the users of software, while API documentation is important for developers who may want to extend the software.

\subsubsection{Communications} 
Providing mechanisms for users to communicate with one another and with the developers can provide another avenue for understanding usage and engagement.

\subsubsubsection{Feedback mechanisms}
One very helpful method of obtaining software usage metrics is to have users directly provide feedback. Providing such feedback mechanisms helps to identify software weaknesses. However, few users provide feedback and often feedback requires interpretation. Individuals are also unlikely to provide positive feedback. Nevertheless, user feedback mechanisms can be powerful for gaining insight from users. 
Feedback mechanisms can be passive or active in nature. More passive mechanisms may involve providing an email address on a website or simply allowing users to post an issue on a GitHub repository. More active mechanisms may include usability testing interviews, Google forms or other surveys, or providing automated GitHub issue templates to encourage specific kinds of feedback.


\subsubsection{Email and support forums}
While it’s possible to collect user feedback and provide support through email,  if a public support forum is used instead, users can learn from answers to questions of other users, reducing the burden on developers, and provide another opportunity for easier tracking of software engagement. This also provides an opportunity to build a community around the software and to learn what is and isn’t working for users. Some forum platforms support upvoting (where other forum users can upvote questions or feature requests). Using an existing support forum platform such as Discourse (\url{https://www.discourse.org/}) provides many features that are convenient in forum moderation, such as user trust levels, categorization and tagging of topics, and detailed tracking of forum posts, forum post views, and user and moderator activity levels that serve as useful metrics about a software project and the community around it.

Engaging with users about what new features they are excited about and acting on those discussions, is a great way to get informed feedback while rewarding user involvement. To further support community, developers could consider inviting user to attend workshops on new features or inviting them to help teach workshops on fundamentals.  Forum members who provide intellectual contributions to projects, including technical support, feature suggestions, highly referenced posts, etc, should be included as authors on relevant papers. And public facing activity summaries (e.g., \url{https://forum.qiime2.org/u/gregcaporaso/summary}) can also be useful for resume building for early career contributors.

Emailing newsletters to registered users, can be a useful method for informing users about new features, updates or issues. Systems like Mailchimp \cite{mailchimp} or HubSpot \cite{hubspot} can allow analytics about how often recipients open the news letter, click on links, or unsubscribe.

\subsubsection{Usability testing}

Usability refers to the ease of use for an individual to use software. Usability testing is a method of purposefully investigating the usability of a software tool. Usability testing can be highly powerful for collecting valuable metrics and information regarding the usage. This involves applying the scientific method to investigating a user's experience with a software tool and asking users to use the software tool while the tester observes their interactions with the tool. 

Often research institutions do not have these experts in user design and testing on staff. However, developers can and should utilize these techniques to conduct their own informal usability testing. Step-by-step guides of how to conduct usability testing have been published elsewhere \cite{savonen_2021}.  It should be noted that the benefits of usability testing are high even when only a few users are observed. 

\subsubsection{Workshops}
Hands-on workshops (online\cite{dillon_experiences_2021} or in-person) for your software are an excellent way to build a user community and get feedback. Workshops allow developers identify what confuses or challenges users. This can be illuminating as the challenges are often not what developers expect. Attendees can also be asked to participate in surveys about what they like and don't like to help guide future development. The quantity, duration, and attendance at your workshops are metrics that can be reported to funding agencies in grant proposals or reports. Posting recordings of events can be shared on YouTube or other platforms, which allows for other useful metrics. 

\subsubsection{Code of conduct}
Before engaging with user or developer communities, it is important to develop and publish a code of conduct for your project to outline expectations of behavior and how contributors can report violations. A code of conduct can be reassuring of community health and are required by some scientific software funders, such as the Chan Zuckerberg Initative’s Essential Open Source Software for Science program. Thus the presence of a Code of Conduct may be a measure that is assessed by potential funders. A good starting point for a code of conduct is the Contributor Covenant (https://www.contributor-covenant.org/), which can be adapted. Adopting and effectively managing\cite{aurora_how_2019} a  Code of Conduct can support the growth of your community by indicating a safe space, thereby encouraging engagement by individuals who might feel intimidated about getting publicly involved.

\subsubsection{Social media}
Having a social media presence through platforms such as twitter, instagram, and youtube, provides opportunities to track engagement with social media posts.  Pairing this with evaluations of engagement with the software itself, to determine if outreach strategies are successful. This can also be helpful to determine if documentation resources are useful. For example, the number of video views on a youtube documentation video can be informative to know what percentage of users may have actually seen the documentation. Videos with many views can also reassure others about using the software . 

\subsection{Reviews}
There are review mechanisms that can help reassure users about software. For example, SourceForge\cite{sourceforge} a platform for publishing and developing software allows users to rate software on the platform. Developers can integrate their GitHub repositories with SorceForge to take advantage of this review platform. Alternatively, GitHub also has a system of adding stars or followers to repositories, however, this appears to be somewhat inconsistently done in the community. 

