Another situation in which metrics such as "registered" users may be distorted, is when a single instance of software is registered for an institution and then a provisioning system allows use for multiple users through for example virtual machines. 

\subsection{Usage metrics assessments}
Metrics should be used to identify how well a software projects is meeting its goal. Once a project is generating sufficient user metrics, the next step is to interpret them to determine how they should inform the project's feature roadmap, development priorities, and outreach initiatives. Interpretation of user metrics can be tricky since any given metric may have many obvious and not so obvious causes. 

The first question a developer may have is how many people are using the tool to begin with? Metrics can be used to estimate the level of "market penetration" that a tool has achieved. Although market share is difficult to estimate, it is important to have an expected community size and a measure of where a tool stands in comparison to its total potential use. It is also important to avoid comparisons between the community size and usage of tools with different approaches, i.e., between a desktop-based genome browser and a web-based repository of cancer data, given the different modes of use. 

When software has sufficient use, observed spikes in usage, both up and down, provide important feedback. A spike may correspond to a class or workshop using the tool or a recent publication that cites the tool. Negative trends in usage may indicate a break in the academic calendar, down time of a host server, or, for tools based on Amazon Web Services, the additional compute resources required by Amazon during holiday cycles may preempt software running on spot instances.

